\section{Introduction}
Automatic image colorization consists in automatically assigning colors to grayscale images, in a plausible and
realistic way, that could potentially fool a human observer. This task is particularly important for advertisement
and film industries, but also in photography and in artist assistance \cite{chromagan}.
In fact, as Yoo et al. underlined in \cite{animation}, "coloring images is one of the most laborious and expensive
stages when making modern day animation movies and comics. Automating the colorization process can help to reduce
both cost and time required in producing comics or animated movies."

Despite many advances in deep learning, automatic image colorization is still nowadays considered a difficult task
and its applications are still limited. This is mainly due to the fact that image colorization is an ill-posed
problem that doesn't have a unique solution \cite{su} \cite{chromagan}. Indeed, from a mathematical point of view,
it requires to map a 2D greyscale image to a 3D colored one, and there are multiple plausible solutions that can
be effectively used to colorize the same object.

In our project we want to explore some of the most recent models for automatic image colorization, such as
ChromaGAN \cite{chromagan}, eccg16 , siggraph17 , instance-aware image colorization \cite{su}. We also used an
older model developed by Dahl \cite{dahl} and we implemented a simple autoencoder as a baseline.



% existing colorization models ignore rare instances present in data and opt to learn the most frequent colors
% to generalize over the data.
% Existing colorization models suffer from the dominant color effect, illustrated in Fig. 2. This effect occurs when
% a colorization model only learns to color with a few dominant colors present in the training set.
% Figure 2. Dominant color effect commonly encountered by deep colorization models. Deep colorization models tend
% to ignore diverse colors present in a training set and opt to learn only a few dominant colors. Using the most
% dominant color can be effective in minimizing the overall loss but yields unsatisfactory re- sults. One can see
% that the outputs of [34] are dominated by the most prevalent color (red).



Introduction (10\%): describe the problem you are working on, why it's important, and an overview of your results.

