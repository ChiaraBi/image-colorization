\documentclass[10pt,twocolumn,letterpaper]{article}

\usepackage{cvpr}
\usepackage{times}
\usepackage{epsfig}
\usepackage{graphicx}
\graphicspath{./img}
\usepackage{amsmath}
\usepackage{amssymb,bbm,xcolor}
\usepackage[breaklinks=true,bookmarks=false]{hyperref}
\usepackage{lipsum}
\usepackage{listings}
\usepackage{mathtools,eucal}
\usepackage{graphicx}

\usepackage{bbold}
\usepackage{caption}
\usepackage{subcaption}
\usepackage{algorithm}
\usepackage[noend]{algpseudocode}
\usepackage{booktabs, multicol, xcolor}
\usepackage[shortlabels]{enumitem}
\usepackage{changepage}
\cvprfinalcopy % *** Uncomment this line for the final submission
\setcounter{page}{1}
\renewcommand{\figurename}{Figure}
\usepackage{amsmath}
\DeclareMathOperator*{\argmax}{arg\,max}
\DeclareMathOperator*{\argmin}{arg\,min}
\DeclareMathAlphabet\mathbfcal{OMS}{cmsy}{b}{n}
\usepackage[T1]{fontenc}
\usepackage{flushend}

% Include other packages here, before hyperref.

% If you comment hyperref and then uncomment it, you should delete
% egpaper.aux before re-running latex.  (Or just hit 'q' on the first latex
% run, let it finish, and you should be clear).
\usepackage[breaklinks=true,bookmarks=false]{hyperref}

\cvprfinalcopy % *** Uncomment this line for the final submission

% Pages are numbered in submission mode, and unnumbered in camera-ready
%\ifcvprfinal\pagestyle{empty}\fi
\setcounter{page}{1}

\begin{document}
\title{Automatic image colorization: a comparative overview}
\author{Bigarella Chiara\\
{\tt\small Student nr. 2004248}
% For a paper whose authors are all at the same institution,
% omit the following lines up until the closing ``}''.
% Additional authors and addresses can be added with ``\and'',
% just like the second author.
% To save space, use either the email address or home page, not both
\and
Poletti Silvia\\
{\tt\small Student nr. 1239133}
}

\maketitle
%\thispagestyle{empty}

%%%%%%%%% ABSTRACT
\begin{abstract}
   The ABSTRACT is to be in fully-justified italicized text, at the top
   of the left-hand column, below the author and affiliation
   information. Use the word ``Abstract'' as the title, in 12-point
   Times, boldface type, centered relative to the column, initially
   capitalized. The abstract is to be in 10-point, single-spaced type.
   Leave two blank lines after the Abstract, then begin the main text.
   Abstract should be no longer than 300 words.
\end{abstract}

\section{Introduction}
Automatic image colorization consists in automatically assigning colors to greyscale images, in a plausible and
realistic way, that could potentially fool a human observer. This task is particularly important for advertisement
and film industries, but also in photography and in artist assistance \cite{chromagan}.
In fact, as Yoo et al. underlined in \cite{animation}, "coloring images is one of the most laborious and expensive
stages when making modern day animation movies and comics. Automating the colorization process can help to reduce
both cost and time required in producing comics or animated movies."

Despite many advances in deep learning, automatic image colorization is still nowadays considered a difficult task
and its applications are still limited. This is mainly due to the fact that image colorization is an ill-posed
problem that doesn't have a unique solution \cite{su} \cite{chromagan}. Indeed, from a mathematical point of view,
it requires to map a 2D greyscale image to a 3D colored one, and there are multiple plausible solutions that can
be effectively used to colorize the same object.

In our project we want to explore some of the most recent models for automatic image colorization, such as
Eccv16 \cite{zhang}, Siggraph17 \cite{siggraph}, ChromaGAN \cite{chromagan}, InstColorization \cite{su}.
We also used an older model developed by Dahl \cite{dahl} and we implemented a simple model based on a
convolutional autoencoder, as a baseline.
To compare all these models, we performed the classification task on the colorized images and measured the
classification accuracy. Furthermore, we computed some advanced metrics, such as Learned Perceptual Image Patch
Similarity (LPIPS), Peak Signal-to-Noise Ratio (PSNR) and Structural Similarity (SSIM) and performed a short
Turing Test.

Overall, the best performing models for what concerns the quantitative metrics are the most recent ones,
i.e. ChromaGAN and InstColorization, as we expected. However, in the Turing Test the best performing models
are ChromaGAN and Siggraph17, while InstColorization performed poorly.



% existing colorization models ignore rare instances present in data and opt to learn the most frequent colors
% to generalize over the data.
% Existing colorization models suffer from the dominant color effect, illustrated in Fig. 2. This effect occurs when
% a colorization model only learns to color with a few dominant colors present in the training set.
% Figure 2. Dominant color effect commonly encountered by deep colorization models. Deep colorization models tend
% to ignore diverse colors present in a training set and opt to learn only a few dominant colors. Using the most
% dominant color can be effective in minimizing the overall loss but yields unsatisfactory re- sults. One can see
% that the outputs of [34] are dominated by the most prevalent color (red).


\section{Related Work}
Related Work (10\%): discuss published work or similar apps that relates to your project. How is your approach similar or different from others?

Papers are: \cite{chromagan}, \cite{su}, \cite{zhang},\cite{siggraph}, \cite{dahl},\cite{animation},\cite{cartoonize},\cite{language}.

\section{Dataset}
We considered three types of images: 4023 originally colored images from five different datasets, 18 originally black and white images from various artists and 180 filtered images (see more details in Experiments section) obtained starting from 18 originally colored images.

Indeed, our data includes heterogeneous images, representing many different environments, situations and subjects.
For what concerns the originally colored images, we considered various sources:
\begin{itemize}
	\item a subset of ImageNet made of 12 classes (200 images each) taken from \cite{imagenette}, ten of which are easily classified classes (tench, English springer, cassette player, chain saw, church, French horn, garbage truck, gas pump, golf ball and parachute) while the other two are not so easy to classify (Samoyed and Rhodesian ridgeback);
	\item a subset of 100 randomly selected images from Pascal VOC \cite{pascal} representing realistic scenes in which the subjects could be animals, human beeings, plants, rooms, landscapes, various objects and vehicles;
	\item a subset of 200 randomly selected images form Places205 \cite{place} reguarding mountain, desert, sea, beach and island landscapes.
	\item a subset of 325 Bird Species \cite{bird} made of 8 classes (100 images each), which were selected to depict those birds having the most unusual colors (Cuban Tody, Fire Tailed Myzornis, Flamingo, Nicobar Pigeon and Pink Robin) and those that are well-known by the majority of people (Bald Eagle, Ostrich and Touchan);
	\item a subset of 102 Category Flowers \cite{flower} made of 6 classes (from 50 to 100 images each), which were selected to depict those flowers having the most unusual colors and shapes (Purple Coneflower, Grape Hyacinth, Hibiscus) and those that are well-known by the majority of people (Rose, Water Lily and Giant White Arum Lily).
\end{itemize}

The images have been preprocessed by using OpenCV (ChromaGAN and InstColorization) or Pillow combined with Skimage
(Baseline, Dahl, Zhang, Siggraph).

The images have been reshaped to various formats ($256\times256\times3$ for Baseline, Zhang, Siggraph and InstColorization and $224\times224\times3$ for Dahl and ChromaGAN) and Dahl also required center cropping and desaturation. Despite the preliminar reshape, Zhang, Siggraph and ChromaGAN models are built in a way that allows to obtain colorized images having the original shape.

Given an RGB image (additive colour model in which red, green and blue primary colour channels are added together)
we obtain the corrisponding image in the \textit{Lab} color space, in which colors are expressed through 3 new
channels: $L$ for perceptual lightness ($L=0$ corresponds to white, $L=100$ corresponds to black), $a$ and $b$ for
four primary colors ($a=\pm100$ correspond to red and green, $b=\pm100$ correspond to yellow and blue). Our models get only the $L$ channel as input (greyscale images) with the goal of predicting the $a$ and $b$
channels. Then, the resulting images are projected again in the RGB color space.

Moreover, the classification with AlexNet required the normalization of the images' RGB channels in the
range $[0,1]$ and a further standardization of the images according to the mean and standard deviation of the
training set images. On the other hand, the LPIPS metric required the normalization of the images' RGB channels in the range $[-1,1]$ and
the dataset reshaping from $N\times H\times W\times3$ to $N\times3\times H\times W$, where $N$ is the number of images.

\section{Methods}
In order to carry out a comparative overview about automatic image colorization, we built, trained and tested a simple autoencoder based on cartoonization, to be considered as baseline. Then, we tested some state-of-the-art pre-trained models taken from the literature: Dahl, Zhang and its upgraded version Siggraph, ChromaGAN and InstColorization.

\subsection{Baseline}
As a baseline, we built with Keras a simple autoencoder having 8 Convolutional layers for the encoding part
(ReLU activations, zero-padding, $3\times3$ kernels and sometimes $2\times2$ strides), while the decoding part
consists in the combination of 5 Convolutional layers (ReLU activations except for the last layer, zero-padding
and $3\times3$ kernel) and 3 UpSampling layers of size $2\times2$. The encoder learns a compact representation
of the black and white input image and the decoder generates the corresponding novel coloured image.

The model was trained (50 epochs) on a heterogeneous dataset containing all the data available.

Moreover, we enriched this model with a novel approach: instead of using the original dataset, we fed the model
with the cartoonized (black and white) version of the images, computed with the pre-trained GAN-based cartoonization
model by \cite{cartoonize}. This cartoonization provides fine-grained results (we don't miss much information)
and synthesizes the original images in order to exclude noisy elements that could interfere with the colorization
task. The model produces cartoonized colored images whose $a$ and $b$ channels are combined with the $L$ channel
of the original \textit{Lab} images. Therefore, we mantain the original details of the pictures, while producing a
more precise and sectorial colorization.

For comparison, we also include in our experiments the Baseline without cartoonization (Baseline w/c).
\subsection{Dahl}

Dahl's model consists in an autoencoder from black and white images to colored ones, with
residuals connections.
As shown in Figure \ref{fig:dahl}, the encoder is a VGG-16 network with ReLU activations, that takes in input greyscale images and infers
some color information at each layer. This information is added up in the decoder thanks to the residual
connections, until a $224 \times 224 \times 3$ tensor is constructed. In the last layer of the decoder, the sigmoid activation is used to squash the values between 0 and 1.

Finally, the model's loss function is the average of the following three Euclidean Distances:
\begin{itemize}
    \item the Euclidean Distance computed between the original colored image and the network output;
    \item the Euclidean Distance computed between the original colored image and the network output, blurred with a 3 pixel gaussian kernel;
    \item the Euclidean Distance computed between the original colored image and the network output, blurred with a 5 pixel gaussian kernel.
\end{itemize}

Differently to the other models, Dahl's model uses the YUV color encoding system which consists of one luma
component (Y) and two chrominance components, called U (blue projection) and V (red projection) respectively.

The biggest disadvantages of this model are the desaturated results and the possible crop and reshape during the preprocessing, leading to some information loss.

% It has been trained on the ILSVRC 2012 classification training dataset.


\subsection{Eccv16}
The innovation introduced by Zhang's colorization model is not the model's architecture (a CNN made of 8 blocks of
two or three repeated Convolutional and ReLU layers followed by a BatchNorm layer, as shown in Figure \ref{fig:zh}) but rather a more suitable loss function for saturated colorization, combined with class rebalancing, which allows to increase the diversity of colors in the results.

Since an object can potentially have several plausible colorization, the model accounts for the multimodal
distribution of possible colors for each pixel. Indeed, the intrinsic multimodal nature of the colorization
problem can't be captured by a simple Euclidean loss between the target and the predicted colors. Instead, the
model learns a map $\mathcal{G}: \mathbb{R}^{H\times W\times1}\rightarrow [0,1]^{H\times W\times Q}$ from the grayscale input to a probability
distribution $\hat{Z}=P(a,b)$ over $Q=313$ possible $(a,b)$ pairs (i.e. colors), which were obtained throug the
quantization of the \textit{ab} output space, as shown in Figure \ref{fig:q}. Then, the multimomial crossentropy
los is defined as:

\begin{equation*}
	\mathcal{L}_{cl} (\hat{Z},Z)= - \sum_{h,w} v(Z_{h,w})\sum_q Z_{h,w,q}log(\hat{Z}_{h,w,q})
\end{equation*}
where $Z$ is the soft-encoded target (obtained by taking the 5-nearest neighbors in the quantized \textit{ab} space
for each groundtruth pixel, and weighting them according to their distance from the groundtruth) and $v$ is a
weighting function for class rebalancing, in order to emphasize rare colors.

To conclude, the final predicted colorization $\hat{Y}$ is the annealed-mean of the distribution $\hat{Z}$, which
consists in taking the mean of the softmax distribution $\sigma_T(\hat{Z}) = \sigma(\hat{Z}/T)$ adjusted according the
temperature parameter T. This avoids desaturated or spatially inconsistent results.
\subsection{ChromaGAN}
The strength of ChromaGAN is to use the semantic understanding of the depicted scene combined with a generative adversarial network (GAN). In fact, the semantic class distribution learning makes ChromaGAN capable of variability (it can provides different colors for objects belonging to the same category, as it happens in reality) while the generative adversarial learning leads to vivid and vibrant colorizations.

The generator $\mathcal{G}_\theta$ is divided into two jointly trained subnetworks: the first one outputs the chrominance information $\mathcal{G}_{\theta_1}^1(L) = (a,b)$ (Figure \ref{fig:chr} in blue) and the second one is a classification network giving in output the class distribution vector $\mathcal{G}_{\theta_2}^2(L)=y$ (Figure \ref{fig:chr} in grey) that is trained to be close to the VGG-16 output, in order to generate useful information for the colorization process. The inital layers (Figure \ref{fig:chr} in yellow) are shared and initialized with the pre-trained VGG-16 weights. Then, both the subnetworks split into two tracks (Figure \ref{fig:chr} in purple and red for $\mathcal{G}_{\theta_1}^1$, and in red and grey for $\mathcal{G}_{\theta_2}^2$). The results are fused by concatenation and used to generate the colors.

The discriminator $\mathcal{D}_w$ focuses on the local patches of the generated image and classifies each of them as real or fake. The ultimate goal is to find the optimum of:
\begin{equation*}
	\min_{\mathcal{G}_\theta}\max_{\mathcal{D}_w} \mathcal{L}(\mathcal{G}_\theta, \mathcal{D}_w) = \mathcal{L}_e(\mathcal{G}^1_{\theta_1}) + \lambda_g\mathcal{L}_g(\mathcal{G}^1_{\theta_1},\mathcal{D}_w) + \lambda_s\mathcal{L}_s(\mathcal{G}^2_{\theta_2})
\end{equation*} 

where $\mathcal{L}_e$ is the expectation of the Euclidean distance between the colorization and the real colors, $\mathcal{L}_s$ is the expectation of the Kullback-Leibler divergence of the predicted class distribution and the VGG-16 pre-trained class distribution, both computed on the grayscale images, and $\mathcal{L}_g$ is an adversarial loss. Note that backpropagation with respect to $\mathcal{L}_s$ only affects $\mathcal{G}_{\theta_2}^2$, while backpropagation with respect to $\mathcal{L}_e$ affects the whole network.
\subsection{InstColorization}
Instead of just performing learning and colorization on the entire image, InstColorization learns meaningful object-level semantics within the bounding boxes localized by an object detector. Then, we have two colorization networks: the first colorizes the whole image and the second the patches (resized to $256\times265$) in the bounding boxes. These networks have different weights but share the same architecture: the chosen architecture is the same as Zhang, as well as the loss function. Once the first networks is trained, its learned weights are used to inizialize the second network. At the end of the second network's training, the resulting full-image features and object-level features have to be combined in a consistent way by a fusion module. This allows to obtain better results on scenes with multiple objects in a cluttered background. The whole process is reported in Figure \ref{fig:su1}.

In particular, the fusion module (Figure \ref{fig:su2}) takes place at multiple layers of the colorization networks. For each layer, the full-image feature and the $N$ object-level features ($N$ is the number of detected objects) are processed by a small CNN and then combined by taking the weighted sum of the stack composed by the full-image weight map and the patches' weight maps, which have been previously reshaped (and zero padded) using the size and location of the bounding boxes for each object.




\section{Experiments}
To compare the results of each model, we computed several metrics: classification with AlexNet, LPIPS, PSNR and SSIM (all quantitative metrics) and a Turing test on few images (qualitative metric). Finally, we applied image filtering to evaluate possible improvements in the performances.

\subsection{Classification with AlexNet}
First, we considered the AlexNet classifier pre-trained on ImageNet and tested on the ImageNet subset in its original, black and white and re-colorized versions. Table \ref{tab:pre-trained} reports the AlexNet classification accuracy in this setting and in other two settings that we will discuss later in this section. Note that the Baseline without cartoonization (Baseline w/c) always reaches a slightly worse accuracy than the Baseline combined with cartoonization which can actually improve the colorization performance.

The great gap in the accuracies computed on the original and the black and white versions of the images suggests that colors play an important role in image classification.

The best colorizations according to this experiment are given by ChromaGAN and InstColorization, while the Baseline and Dahl are not even able to improve the accuracy with respect to the black and white images.

Overall, the accuracy on the models' colorizations is much lower than the one computed on the original images and the latter is relatively low. Therefore we applied feature extraction to better focus on our ImageNet subset: we used the pre-trained AlexNet as a fixed feature-extractor, and only updated the final layer (for 2 epochs) in order to consider just our 12 ImageNet classes. This resulted in more reliable accuracy values and all the models except the Baseline are able to outperform the black and white images.

For a further comparison, we applied finetuning to perform classification on the birds and flowers images, which present more vibrant and various colors than our ImageNet subset: we updated (for 2 epochs) all the AlexNet parameters for the new task. In this new setting we have, as expected, a greater gap than before between the original and the black and white accuracies, meaning that the color is much more relevant. Indeed, all the models including the Baseline with cartoonization are able to improve the accuracy with respect to the black and white images.

The best colorizations according to this experiment are given by the Eccv16 and Siggraph17 models, which are able to generalize better across different datasets.

In this last setting, we can notice a general decreasing in the accuracy (except for the original images) with respect to the feature extraction using the ImageNet subset. This is due to the fact that our pre-trained models have been trained on Image-Net and their colorization of the birds and flowers images are overall bad. However, looking at our results, a badly colored image generally seems more distinguishable than its black and white version.

To conclude, the colorizations of two images are reported as an example in Figure \ref{fig:imagenet}.


\subsection{LPIPS, PSNR and SSIM Metrics}


\subsection{Turing Test}
In addition to the aforementioned quantitative metrics, we decided to perform a short Turing test, to use also
a qualitative metric in the models comparison. To do so, we created a short survey on Google Forms. The survey is
made of two parts: in the first part we show the participants three different black and white images, each one of
them followed by the corresponding colorized images; while for the second part we chose 4 different images, which
are originally colored, and we displayed the corresponding colorized images, without showing the original image.
For all the automatically colorized images, we asked the participant to say how realist the colorization was in a
scale that goes from 1 (not realistic at all) to 5 (very realistic). Note that for the Turing Test we employed only images coming from the four best
performing models, i.e. Eccv16, Siggraph17, ChromaGAN and InstColorization.

We collected answers from 124 different subjects and in Figure \ref{fig:turing} we can see the mean scores
obtained by the differnt models. From the plots we can clearly see that the models performed differently on black and white images and on colored
images. This can be explained by the fact that black and white pictures and colored pictures are taken using
different technologies.
However, the models that obtained the higest scores in both cases are Siggraph17 and ChromaGAN.

\begin{figure}[h]
	\centering
	\captionsetup[subfigure]{labelformat=empty}
	\begin{subfigure}[b]{0.1\textwidth}
		\begin{adjustwidth}{-1.1cm}{}
		\includegraphics[width=4cm]{bw turing.png}
		\end{adjustwidth}
	\caption{B\&W}
	\end{subfigure}
\hspace{2.3cm}
	\begin{subfigure}[b]{0.1\textwidth}
		\begin{adjustwidth}{-1.1cm}{}
			\includegraphics[width=4cm]{col turing.png}
		\end{adjustwidth}
		\caption{Colored}
	\end{subfigure}
	\caption{{\small Mean scores obtained with the colorization on black and white photographs and originally colored images.}}
	\label{fig:turing}
\end{figure}
\subsection{Image filtering}
possiamo prendere ad esempio ChromaGAN (uno dei migliori modelli): l'erba la fa verde perchè è più realistica (si potrebbe pensare che ha colori anche migliori dell'originale), cartoon fa schifo perchè il modello non ha fatto un training su tali immagini, altrimenti potremmo supporre che performerebbe meglio. Infatti non riconosce l'erba come tale e sembra la colori di blu, scambiandola per acqua.
Figure \ref{fig:filter}

\begin{figure*}[t]
	\centering
	\captionsetup[subfigure]{labelformat=empty}
		\begin{subfigure}[b]{0.1\textwidth}
		\centering
		\includegraphics[width=2.4cm]{orig - filter.jpeg}
		\caption{Original}
		\end{subfigure}
		\hfill
		\begin{subfigure}[b]{0.1\textwidth}
			\includegraphics[width=2.4cm]{orig - filter - blur.jpeg}
			\caption{Blurred}
		\end{subfigure}
		\hfill
		\begin{subfigure}[b]{0.1\textwidth}
			\includegraphics[width=2.4cm]{orig - filter - cartoon.jpeg}
			\caption{Cartoon}
		\end{subfigure}
		\hfill
		\begin{subfigure}[b]{0.1\textwidth}
			\includegraphics[width=2.4cm]{orig - filter - man contr (2).jpg}
			\caption{Low contrast}
		\end{subfigure}
		\hfill
		\begin{subfigure}[b]{0.1\textwidth}
			\includegraphics[width=2.4cm]{orig - filter - man contr (1).jpg}
			\caption{High contrast}
		\end{subfigure}
		\hfill
		\begin{subfigure}[b]{0.1\textwidth}
			\includegraphics[width=2.4cm]{orig - filter - lumin (1).jpeg}
			\caption{Brighter}
		\end{subfigure}
		\hfill
		\begin{subfigure}[b]{0.1\textwidth}
			\includegraphics[width=2.4cm]{orig - filter - lumin (2).jpeg}
			\caption{Darker}
		\end{subfigure}
	
				\begin{subfigure}[b]{0.1\textwidth}
			\centering
			\includegraphics[width=2.4cm]{c - filter.jpeg}

		\end{subfigure}
		\hfill
		\begin{subfigure}[b]{0.1\textwidth}
			\includegraphics[width=2.4cm]{c - filter - blurr.jpeg}
		\end{subfigure}
		\hfill
		\begin{subfigure}[b]{0.1\textwidth}
			\includegraphics[width=2.4cm]{c - filter - cartoon.jpeg}
		
		\end{subfigure}
		\hfill
		\begin{subfigure}[b]{0.1\textwidth}
			\includegraphics[width=2.4cm]{c - filter - man contr (1).jpg}
	
		\end{subfigure}
		\hfill
		\begin{subfigure}[b]{0.1\textwidth}
			\includegraphics[width=2.4cm]{c - filter - man contr (2).jpg}
	
		\end{subfigure}
		\hfill
		\begin{subfigure}[b]{0.1\textwidth}
			\includegraphics[width=2.4cm]{c - filter - lumin (1).jpeg}
	
		\end{subfigure}
		\hfill
		\begin{subfigure}[b]{0.1\textwidth}
			\includegraphics[width=2.4cm]{c - filter - lumin (2).jpeg}
	
		\end{subfigure}
	\caption{{\small ChromaGAN colorization (second row) on some filtered images (first row).}}
	\label{fig:filter}
\end{figure*}

\section{Conclusion}
As expected, the models having the overall best performances are the state-of-the-art ChromaGAN and Siggraph17. Eccv16 produces vibrant colorizations with the use of class-rebalancing, but at
the expense of some over-aggressively and visibly artificial colorizations. In general, InstColorization is able to improve the results of Eccv16, except for those classes showing bright and rare colors (low accuracy when tested with AlexNet finetuned on birds and flowers images). Moreover, its segmentated colorization can lead to incoherent results when object detection doesn't work properly, and this came out in the Turing test scores. 

We have also seen that some image filtering concerning luminance and contrast can improve the final colorization.

In conclusion, our baseline performs very poorly due to the limited amount of data for the training, the simple
architecture and the absence of a colorization-adapted loss function. However, the introduction of cartoonization
in the training and testing phases seems to slightly improve the final results, meaning that more sectorial and
definite areas are easier to colorize. This opens up new perspectives for future works based on the combination
of state-of-the-art models with cartoonization.


{\small
\bibliographystyle{ieee_fullname}
\bibliography{egbib}
}

\end{document}
