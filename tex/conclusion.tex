\section{Conclusion}
As expected, the model reaching the overall best performances is the state-of-the-art ChromaGAN, followed by Siggraph17. Eccv16 produces vibrant colorizations with the use of class-rebalancing, but at
the expense of some over-aggressively and visibly artificial colorizations. In general, InstColorization is able to improve the results of Eccv16, except for those classes showing bright and rare colors (low accuracy when tested with AlexNet finetuned on birds and flowers images). Moreover, its segmentated colorization can lead to incoherent results when object detection doesn't work properly, and this came out in the Turing test scores. 

We have also seen that some image filtering concerning luminance and contrast can improve the final colorization.

In conclusion, our baseline performs very poorly due to the limited amount of data for the training, the simple
architecture and the absence of a colorization-adapted loss function. However, the introduction of cartoonization
in the training and testing phases seems to slightly improve the final results, meaning that more sectorial and
definite areas are easier to colorize. This opens up new perspectives for future works based on the combination
of state-of-the-art models with cartoonization.
