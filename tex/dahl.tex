\subsection{Dahl}

Dahl's model is an autoencoder from black and white images to colored ones, with
residuals connections.
As shown in Figure \ref{fig:dahl}, the encoder is a VGG-16 network with ReLUs as activation functions, that takes in input greyscale images and infers
some color information at each layer. This information is added up in the decoder thanks to the residual
connections, until a 224 x 224 x 3 tensor is constructed. In the last layer of the decoder, a sigmoid activation
function is used to squash the values between 0 and 1.
Finally, the model's loss function is the average of the following three Euclidean Distances:
\begin{itemize}
    \item the Euclidean Distance computed between the original colored image and the network output;
    \item the Euclidean Distance computed between the original colored image and the network output, blurred with a 3 pixel gaussian kernel;
    \item the Euclidean Distance computed between the original colored image and the network output, blurred with a 5 pixel gaussian kernel.
\end{itemize}

Differently to the other models, Dahl's model uses the YUV color encoding system which consists of one luma
component (Y) and two chrominance components, called U (blue projection) and V (red projection) respectively.

One of the biggest disadvantage of this model is the fact that it handles 224 x 224 images only. This means that
bigger images have to be cropped and reshaped during the preprocessing, leading to some information loss.

% It has been trained on the ILSVRC 2012 classification training dataset.

