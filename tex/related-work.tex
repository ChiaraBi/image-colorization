\section{Related Work}
In the past years, several image colorization models have been proposed and developed. They have been classified into
three main cathegories: scribble-based models, example-based models and deep learning models. In this report we will
focus on the latter cathegory.

Among the deep learning models, a common approach has been to use deep convolutional neural networks based on the
VGG-16 architecture. Four of the pretrained models we used in our comparative analysis are indeed based on this
approach. We are referring to Dahl's model \cite{dahl}, the two models developed by Zhang et al., i.e.
Eccv16 \cite{zhang} and Siggraph17 \cite{siggraph} and the more recent InstColorization \cite{su}. Despite of some
small changes in the models' architectures, the most important difference among those models consists in the loss
functions they use.

Another approach consists in using memory networks, i.e. neural networks augmented with an external
memory module used to store some critical information. An example of this method is MemoPainter \cite{animation},
a model that combines a colorization network with a memory network to store rare instances, thus allowing the
colorization network to produce high-quality colorization with a limited amount of data.

A recent successful approach consists in employing Generative Adversarial Networks (GANs), where a discriminator
tries to discriminate between real and fake images and a generator tries to fool the discriminator by coloring
greyscale images in a realistic way. State-of-the-art models such as ChromaGAN \cite{chromagan} are based on this
approach.

Finally, some models exploit user inputus to improve the colorization process. That's the case of
Siggraph17 \cite{siggraph} and \cite{language}, that use the choice of some colors and words respectively.

The aforementioned methods are just a few examples of the several techniques implied in solving the complex task
of image colorization, but there are many more that we won't cover here.
