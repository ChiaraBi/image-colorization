\subsection{Baseline}
As a baseline, we built with Keras a simple autoencoder having 8 Convolutional layers for the encoding part
(ReLU activations, zero-padding, $3\times3$ kernels and sometimes $2\times2$ strides), while the decoding part
consists in the combination of 5 Convolutional layers (ReLU activations except for the last layer, zero-padding
and $3\times3$ kernel) and 3 UpSampling layers of size $2\times2$. The encoder learns a compact representation
of the black and white input image and the decoder generates the corresponding novel coloured image.

The model was trained (50 epochs) on a heterogeneous dataset containing all the data available.

Moreover, we enriched this model with a novel approach: instead of using the original dataset, we fed the model
with the cartoonized (black and white) version of the images, computed with the pre-trained GAN-based cartoonization
model by \cite{cartoonize}. This cartoonization provides fine-grained informative results 
and synthesizes the original images in order to exclude noisy elements that could interfere with the colorization
task. The model produces cartoonized colored images whose $a$ and $b$ channels are combined with the $L$ channel
of the original \textit{Lab} images. Therefore, we mantain the original details of the pictures, while producing a
more precise and sectorial colorization.

For comparison, we also include in our experiments the Baseline without cartoonization (Baseline w/c).